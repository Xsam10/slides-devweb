\hypertarget{programme}{%
\section{Programme}\label{programme}}

\begin{itemize}
\tightlist
\item
  Frameworks MVC : Laravel, Django, \ldots{}
\item
  HTML5 : vue d'ensemble
\item
  Javascript : VueJS, Node.js, jQuery, AJAX, JSON, \ldots{}
\item
  Déploiement et configuration Serveur
\item
  Webservices : REST vs SOAP
\item
  Sécurité : Technologies, prévention des risques courants
\item
  (Responsive) Web Design
\item
  (Syndication : RSS, Atom)
\item
  {Vos souhaits ?}
\end{itemize}

\hypertarget{contenu-activituxe9s}{%
\section{Contenu, activités}\label{contenu-activituxe9s}}

\begin{itemize}
\tightlist
\item
  Cours théorique
\item
  2 Projets

  \begin{itemize}
  \tightlist
  \item
    2 frameworks : Laravel \& Django (ouvert à d'autres propositions)
  \item
    Groupes de 3, \href{https://www.he-arc.ch/reglementation}{30h} par
    personne et par projet
  \item
    Présentation de 20min
  \end{itemize}
\item
  Workshops intervenants externes

  \begin{itemize}
  \tightlist
  \item
    Webdesign (\href{https://www.alinekeller.ch}{A. Keller}) ?
  \item
    Flask (\href{http://www.matthieuamiguet.ch/}{M. Amiguet}) ?
  \item
    Automatisation du déploiement
    (\href{https://www.linkedin.com/in/raphaelemourgeon/}{R. Emourgeon})
    ?
  \item
    {Vos présentations ? Vos propositions ?}
  \end{itemize}
\item
  Support : \href{https://he-arc.github.io/slides-devweb/}{ghpages}
  (\href{https://github.com/HE-Arc/slides-devweb/tree/master/src}{source}),
  \ldots ProfsAEtudiants\textbackslash3255\_Technologies\_Interaction\textbackslash dw
\end{itemize}

\hypertarget{projets}{%
\section{Projets}\label{projets}}

\begin{itemize}
\tightlist
\item
  Faire pour apprendre
\item
  Les rôles dans une équipe de développement web, workflow
\item
  Ne pas réinventer la roue ou tout faire soi-même
\item
  Critères d'évaluation d'un projet
\item
  En profiter pour apprendre des choses qui vous intéressent
\item
  Avant le 1er octobre :

  \begin{itemize}
  \tightlist
  \item
    Avoir un compte github avec une
    \href{https://github.com/settings/keys}{clé SSH} (indispensable au
    déploiement)
  \item
    Constitution des équipes de 3 personnes
  \item
    Choix du projet
  \item
    Forge : Créer projet sur github dans l'entité
    \href{https://github.com/HE-Arc/}{HE-Arc}
  \item
    \href{https://github.com/HE-Arc/slides-devweb/wiki/Projets-2020-2021}{S'inscrire}
  \end{itemize}
\item
  Offre d'essai Pluralsight 1-6 mois sur
  \href{https://my.visualstudio.com/Benefits}{MS Imagine},
  \href{https://docs.microsoft.com/fr-fr/visualstudio/subscriptions/vs-pluralsight}{infos}
\end{itemize}

\hypertarget{choix-des-projets}{%
\section{Choix des projets}\label{choix-des-projets}}

\begin{itemize}
\tightlist
\item
  Contrainte : appli basée sur des données
\item
  Choix

  \begin{itemize}
  \tightlist
  \item
    Besoin réel
  \item
    Données existantes : \href{http://wiki.dbpedia.org/}{dbpedia},
    \href{https://opendata.swiss/fr/}{opendata}, \ldots{}
  \item
    S'inspirer de l'existant :

    \begin{itemize}
    \tightlist
    \item
      \href{https://www.producthunt.com/topics/web-app}{Product Hunt},
      \href{http://www.makeuseof.com/tag/best-websites-internet/}{makeuseof},
      \ldots{}
    \item
      \href{https://he-arc.github.io}{Volées précédentes}
    \end{itemize}
  \end{itemize}
\item
  Commencer tôt pour se libérer les dernières semaines de l'année
\end{itemize}

\hypertarget{calendrier}{%
\section{Calendrier}\label{calendrier}}

\begin{longtable}[]{@{}rlrl@{}}
\toprule
Semaine & Automne & Semaine & Printemps\tabularnewline
\midrule
\endhead
38 & & 8 &\tabularnewline
39 & Projet PHP & 9 &\tabularnewline
40 & & 10 &\tabularnewline
41 & & 11 &\tabularnewline
43 & & 12 &\tabularnewline
44 & & 13 &\tabularnewline
45 & & 15 &\tabularnewline
46 & & 16 & Présentations\tabularnewline
47 & S. thématique & 17 & Présentations\tabularnewline
48 & & 18 & Présentations\tabularnewline
49 & & 19 &\tabularnewline
50 & & 20 & Examens\tabularnewline
51 & Présentations & 21 & Début TB\tabularnewline
2 & Projet Python & &\tabularnewline
3 & & &\tabularnewline
4 & & &\tabularnewline
5 & T. Autonome & &\tabularnewline
6 & Examen & &\tabularnewline
\bottomrule
\end{longtable}

\hypertarget{suivi-du-calendrier-uxe0-jour-sur-le-ruxe9seau-interne}{%
\section{Suivi du calendrier (à jour sur le réseau
interne)}\label{suivi-du-calendrier-uxe0-jour-sur-le-ruxe9seau-interne}}

\begin{figure}
\centering
\includegraphics{src/img/DW2021.png}
\caption{Suivi calendrier}
\end{figure}

\hypertarget{jalons-pour-chacun-des-2-projets}{%
\section{Jalons pour chacun des 2
projets}\label{jalons-pour-chacun-des-2-projets}}

\begin{itemize}
\tightlist
\item
  Echéances

  \begin{itemize}
  \tightlist
  \item
    En début de semaine, pour chacun des projets :

    \begin{enumerate}
    \def\labelenumi{\arabic{enumi}.}
    \tightlist
    \item
      Formation équipe et choix thème
    \item
      Objectifs et maquettes
    \item
      Authentification et 1er déploiement
    \item
    \item
      Modèles avec relations (au moins 3, dont 1 n-n)
    \item
    \item
      Minimal Viable Product
    \item
    \item
    \item
    \item
      Rendu projet, Présentation
    \end{enumerate}
  \end{itemize}
\item
  Il n'est pas interdit d'en ajouter
\end{itemize}

\hypertarget{conseils}{%
\section{Conseils}\label{conseils}}

\begin{itemize}
\tightlist
\item
  Le plus simple possible
\item
  Pas trop de données
\item
  Application crédible (vraies données, cas réalistes)
\item
  Projet à blanc pour la prise en main du framework
\item
  \href{https://brainhub.eu/blog/difference-between-wireframe-mockup-prototype/}{Maquettes}
\item
  \href{http://drewfradette.ca/a-simpler-successful-git-branching-model/}{Organisez}
  l'utilisation du dépôt
\item
  Le temps disponible à l'horaire ne suffira pas !
\item
  Essayez de commit avec la même identité
\item
  Signalez dans le commit msg si vous n'êtes pas l'auteur
\item
  Le déploiement est long : commencez tôt !
\item
  Il est moins risqué travailler plus au début du projet qu'à la fin !
\item
  Discutez ! Echangez ! par exemple avec les vieux sur
  \href{https://gitter.im/HE-Arc}{gitter}
\end{itemize}

\hypertarget{uxe9valuation-des-projets}{%
\section{Évaluation des projets}\label{uxe9valuation-des-projets}}

\begin{itemize}
\tightlist
\item
  Note finale d'un projet :

  \begin{itemize}
  \tightlist
  \item
    User Experience : 50\%

    \begin{itemize}
    \tightlist
    \item
      Design UI, Utilisabilité (Efficacité, efficience, satisfaction)
    \end{itemize}
  \item
    Code : 30\%

    \begin{itemize}
    \tightlist
    \item
      Absence bugs, qualité code, lisibilité, respect conventions et
      bonnes pratiques
    \item
      Déploiement, configuration
    \end{itemize}
  \item
    Gestion de projet : 20\%

    \begin{itemize}
    \tightlist
    \item
      Fichiers versionnés, messages de commit, Issues, planification,
      travail en équipe
    \item
      Documentation (wiki), Investissement, volume de travail
    \end{itemize}
  \item
    Bonus (ceux qui vont plus loin) : 0-20\%

    \begin{itemize}
    \tightlist
    \item
      WebSockets ou autre API HTML5, webservices, \ldots{}
    \item
      Contribution, présentation, documentation, \ldots{}
    \end{itemize}
  \end{itemize}
\item
  {Tous les membres d'un groupe n'ont pas forcément la même note}
\end{itemize}

\hypertarget{participation}{%
\section{Participation}\label{participation}}

\begin{itemize}
\tightlist
\item
  Aux projets des autres : Issues, PR
\item
  Participez à la
  \href{https://hacktoberfest.digitalocean.com/}{Hacktoberfest}
\item
  Pariticipez au cours : contenu, présentation, pages (index, wiki,
  \ldots)
\end{itemize}

\hypertarget{pruxe9sentation-facultative}{%
\section{Présentation facultative}\label{pruxe9sentation-facultative}}

\begin{itemize}
\tightlist
\item
  Facultatif, ne peut qu'augmenter la moyenne
\item
  DOIT être annoncé au semestre d'automne
\item
  Un thème absent du cours
\item
  2 à 4 personnes
\item
  Une présentation claire avec démo (printemps)
\item
  Un exercice d'application
\item
  Critiques et discussion
\item
  Au plus tôt :

  \begin{itemize}
  \tightlist
  \item
    Constitution des équipes
  \item
    Proposer 1 à 3 thèmes
  \item
    \href{https://docs.google.com/spreadsheet/viewform?formkey=dEVJRE1WVTVPelhFcE94TGF5N1c0cGc6MQ}{Proposer}
    le(s) thème(s) de présentation et l'équipe
  \end{itemize}
\end{itemize}

\hypertarget{examen-oral-sa}{%
\section{Examen oral SA}\label{examen-oral-sa}}

\begin{itemize}
\tightlist
\item
  Généralités pour la partie dev web de l'examen :

  \begin{itemize}
  \tightlist
  \item
    Vous n'avez droit à rien : on vous mettra à disposition 1 crayon et
    du papier pour préparer votre présentation,
  \item
    L'examen porte sur toute la matière vue au en cours (yc workshops),
  \item
    Les questions sont générales, il s'agit de présenter des concepts
    vus en cours (souvent 1 chapitre), et expliquer certains mécanismes
    sous-jacents,
  \item
    Il n'agit pas de réciter le contenu des slides par coeur, mais de
    les présenter avec vos propres mots (compréhension), et vos propres
    exemples.
  \end{itemize}
\item
  Généralités pour la partie dev mobile de l'examen :

  \begin{itemize}
  \tightlist
  \item
    Vous pourriez avoir un résumé personnel manuscrit, d'une feuille A4
    recto-verso,
  \item
    L'examen porte sur toute la matière vue en cours,
  \item
    Les questions peuvent être théoriques ou/et pratiques.
  \end{itemize}
\end{itemize}

\hypertarget{examen-oral-sa-1}{%
\section{Examen oral SA}\label{examen-oral-sa-1}}

\begin{itemize}
\tightlist
\item
  Déroulement :

  \begin{itemize}
  \tightlist
  \item
    Vous tirez un n° de question au hasard pour chaque cours
  \item
    Vous disposez de 15 min pour préparer une présentation de 10 min
    pour chacun des 2 cours (pendant la présentation de l'étudiant
    précédent)
  \item
    Idéalement vous faites une présentation d'environ 10 min et les 5
    min restantes sont dédiées aux questions (pour chacun des cours)
  \end{itemize}
\end{itemize}

\hypertarget{mesures-sanitaires-covid-19}{%
\section{Mesures sanitaires
COVID-19}\label{mesures-sanitaires-covid-19}}

\begin{itemize}
\tightlist
\item
  Rentrée : 100\% présentiel
\item
  Etudiants portent le masque
\item
  Enseignants portent le masque à moins d'1.50m
\item
  Désinfection des mains avant d'entrer en salle
\item
  Quarantaine :

  \begin{itemize}
  \tightlist
  \item
    courte en cas de suspicion : le temps d'avoir le résultat des tests
    (2-5j)

    \begin{itemize}
    \tightlist
    \item
      considérée comme une absence, nécessité d'un certificat, faire le
      nécessaire pour rattrapper
    \end{itemize}
  \item
    longue (14 jours) si test positif
  \item
    quarantaine courte ou des enseignants : le cours sera donné via
    Teams si possible
  \end{itemize}
\end{itemize}

\hypertarget{mon-expuxe9rience-en-duxe9veloppement-web}{%
\section{Mon expérience en développement
web}\label{mon-expuxe9rience-en-duxe9veloppement-web}}

\begin{itemize}
\tightlist
\item
  \href{https://docs.google.com/spreadsheet/viewform?formkey=dDg5Znh5akRBV1hPbC1qYlVRV3BONFE6MQ}{Questionnaire}
  obligatoire (votre username github vous y sera demandé)
\end{itemize}

\hypertarget{m-e-r-c-i}{%
\subsubsection{M E R C I !}\label{m-e-r-c-i}}

\hypertarget{sources}{%
\section{Sources}\label{sources}}
